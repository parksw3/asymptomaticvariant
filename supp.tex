\documentclass[12pt]{article}
% \usepackage[top=1in,left=1in, right = 1in, footskip=1in]{geometry}
\usepackage[top=1in,footskip=1in]{geometry}

\usepackage{graphicx}
\usepackage{xspace}
%\usepackage{adjustbox}

\usepackage{grffile}

\newcommand{\comment}{\showcomment}
%% \newcommand{\comment}{\nocomment}

\newcommand{\showcomment}[3]{\textcolor{#1}{\textbf{[#2: }\textsl{#3}\textbf{]}}}
\newcommand{\nocomment}[3]{}

\newcommand{\jd}[1]{\comment{cyan}{JD}{#1}}
\newcommand{\swp}[1]{\comment{magenta}{SWP}{#1}}
\newcommand{\weitz}[1]{\comment{blue}{WEITZ}{#1}}

\newcommand{\eref}[1]{Eq.~(\ref{eq:#1})}
\newcommand{\fref}[1]{Fig.~\ref{fig:#1}}
\newcommand{\Fref}[1]{Fig.~\ref{fig:#1}}
\newcommand{\sref}[1]{Sec.~\ref{#1}}
\newcommand{\frange}[2]{Fig.~\ref{fig:#1}--\ref{fig:#2}}
\newcommand{\tref}[1]{Table~\ref{tab:#1}}
\newcommand{\tlab}[1]{\label{tab:#1}}
\newcommand{\seminar}{SE\mbox{$^m$}I\mbox{$^n$}R}

\usepackage{amsthm}
\usepackage{amsmath}
\usepackage{amssymb}
\usepackage{amsfonts}
\usepackage[utf8]{inputenc} % make sure fancy dashes etc. don't get dropped

\usepackage{lineno}
\linenumbers

\usepackage[pdfencoding=auto, psdextra]{hyperref}

\usepackage{natbib}
\bibliographystyle{unsrt}
\date{\today}

\usepackage{xspace}
\newcommand*{\ie}{i.e.\@\xspace}

\usepackage{color}

\newcommand{\Rx}[1]{\ensuremath{{\mathcal R}_{#1}}\xspace} 
\newcommand{\RR}{\ensuremath{{\mathcal R}}\xspace}
\newcommand{\Rres}{\Rx{\mathrm{res}}}
\newcommand{\Rinv}{\Rx{\mathrm{inv}}}
\newcommand{\Rhat}{\ensuremath{{\hat\RR}}}
\newcommand{\Rt}{\ensuremath{{\mathcal R}(t)}\xspace}
\newcommand{\tsub}[2]{#1_{{\textrm{\tiny #2}}}}
\newcommand{\dd}[1]{\ensuremath{\, \mathrm{d}#1}}
\newcommand{\dtau}{\dd{\tau}}
\newcommand{\dx}{\dd{x}}
\newcommand{\dsigma}{\dd{\sigma}}

\newcommand{\rx}[1]{\ensuremath{{r}_{#1}}\xspace} 
\newcommand{\rres}{\rx{\mathrm{res}}}
\newcommand{\rinv}{\rx{\mathrm{inv}}}

\newcommand{\psymp}{\ensuremath{p}} %% primary symptom time
\newcommand{\ssymp}{\ensuremath{s}} %% secondary symptom time
\newcommand{\pinf}{\ensuremath{\alpha_1}} %% primary infection time
\newcommand{\sinf}{\ensuremath{\alpha_2}} %% secondary infection time

\newcommand{\psize}{{\mathcal P}} %% primary cohort size
\newcommand{\ssize}{{\mathcal S}} %% secondary cohort size

\newcommand{\gtime}{\tau_{\rm g}} %% generation interval
\newcommand{\gdist}{g} %% generation-interval distribution
\newcommand{\idist}{\ell} %% incubation-period distribution

\newcommand{\total}{{\mathcal T}} %% total number of serial intervals

\usepackage{lettrine}

\newcommand{\dropcapfont}{\fontfamily{lmss}\bfseries\fontsize{26pt}{28pt}\selectfont}
\newcommand{\dropcap}[1]{\lettrine[lines=2,lraise=0.05,findent=0.1em, nindent=0em]{{\dropcapfont{#1}}}{}}

\begin{document}

\begin{flushleft}{
	\Large
	\textbf\newline{
		Supplementary Materials for Intermediate levels of asymptomatic transmission can lead to the highest epidemic fatalities
	}
}
\newline
\\
Sang Woo Park\textsuperscript{1},
Jonathan Dushoff\textsuperscript{2,3,4},
Bryan T. Grenfell\textsuperscript{1,5},
Joshua S. Weitz\textsuperscript{6,7,8,*}
\\
\bigskip
\textbf{1} Department of Ecology and Evolutionary Biology, Princeton University, Princeton, NJ, USA
\\
\textbf{2} Department of Biology, McMaster University, Hamilton, ON, Canada
\\
\textbf{3} Department of Mathematics and Statistics, McMaster University, Hamilton, ON, Canada
\\
\textbf{4} M.\,G.\,DeGroote Institute for Infectious Disease Research, McMaster University, Hamilton, ON, Canada
\\
\textbf{5} Princeton School of Public and International Affairs, Princeton University, Princeton, NJ, USA
\\
\textbf{6} School of Biological Sciences, Georgia Institute of Technology, Atlanta, GA, USA
\\
\textbf{7} School of Physics, Georgia Institute of Technology, Atlanta, GA, USA
\\
\textbf{8} Institut de Biologie, \'{E}cole Normale Sup\'{e}rieure, Paris, France
\\
\bigskip

*Corresponding author: jsweitz@gatech.edu
\bigskip
\end{flushleft}

\setcounter{figure}{0}
\renewcommand{\thefigure}{S\arabic{figure}}
\renewcommand{\thetable}{S\arabic{table}}

\pagebreak

\section*{Supplementary Tables}

\begin{table}[h!]
  \begin{center}
    \begin{tabular}{c|l|c} % <-- Alignments: 1st column left, 2nd middle and 3rd right, with vertical lines in between
      \textbf{Parameter} & \textbf{Description} & \textbf{Assumed values}\\
      \hline
      $\beta_s$ & Symptomatic transmission rate & $0.8/\mathrm{days}$\\
      \hline
      $\beta_a$ & Asymptomatic transmission rate & $0.75 \beta_s$ \\
      \hline
      $1/\nu$ & Mean latent period & $2\ \mathrm{days}$\\
      \hline
      $1/\gamma_s$ & Mean symptomatic infectious period & $5\ \mathrm{days}$\\
      \hline
      $1/\gamma_a$ & Mean asymptomatic infectious period & $5\ \mathrm{days}$\\
      \hline
      $p$ & Proportion asymptomatic & 0--1\\
      \hline
      $f$ & Fatality rate for symptomatic case & 0.01\\
      \hline
      $\delta$ & Reduction in symptomatic transmission rate & 0--1\\
    \end{tabular}
    \caption{Parameter descriptions and values for the basic asymptomatic model.}
    \label{tab:table1}
  \end{center}
\end{table}

\pagebreak

\begin{table}[h!]
  \begin{center}
    \begin{tabular}{c|l|c} % <-- Alignments: 1st column left, 2nd middle and 3rd right, with vertical lines in between
      \textbf{Parameter} & \textbf{Description} & \textbf{Assumed values}\\
      \hline
      $\beta_s$ & Symptomatic transmission rate & See Materials and Methods\\
      \hline
      $\beta_a$ & Asymptomatic transmission rate & See Materials and Methods \\
      \hline
      $\beta_p$ & Presymptomatic transmission rate & See Materials and Methods \\
      \hline
      $1/\nu$ & Mean latent period & $2\ \mathrm{days}$\\
      \hline
      $1/\sigma$ & Mean presymptomatic infectious period & $2\ \mathrm{days}$\\
      \hline
      $1/\gamma_s$ & Mean symptomatic infectious period & $3\ \mathrm{days}$\\
      \hline
      $1/\gamma_a$ & Mean asymptomatic infectious period & $3\ \mathrm{days}$\\
      \hline
      $p$ & Proportion asymptomatic & 0--1\\
      \hline
      $f$ & Fatality rate for symptomatic case & 0.01\\
      \hline
      $\delta_s$ & Reduction in symptomatic transmission rate & 0--1\\
    \end{tabular}
    \caption{Parameter descriptions and values for the generalized asymptomatic model.}
    \label{tab:table2}
  \end{center}
\end{table}

\pagebreak

\begin{table}[h!]
  \begin{center}
    \begin{tabular}{c|l|c} % <-- Alignments: 1st column left, 2nd middle and 3rd right, with vertical lines in between
      \textbf{Parameter} & \textbf{Description} & \textbf{Assumed values}\\
      \hline
      $\beta_s$ & Symptomatic transmission rate & $0.8/\mathrm{days}$\\
      \hline
      $\beta_a$ & Asymptomatic transmission rate & $0.75 \beta_s$ \\
      \hline
      $1/\nu$ & Mean latent period & $2\ \mathrm{days}$\\
      \hline
      $1/\gamma_s$ & Mean symptomatic infectious period & $5\ \mathrm{days}$\\
      \hline
      $1/\gamma_a$ & Mean asymptomatic infectious period & $5\ \mathrm{days}$\\
      \hline
      $p$ & Proportion asymptomatic & 0--1\\
      \hline
      $f$ & Fatality rate for symptomatic case & 0.01\\
      \hline
      $\delta$ & Reduction in symptomatic transmission rate & 0--1\\
      \hline
      $\epsilon_i$ & Protection against infection & 0--0.8\\
      \hline
      $\epsilon_s$ & Protection against symptoms & 0--0.8\\
      \hline
      $\epsilon_d$ & Protection against deaths & 0--0.8\\
    \end{tabular}
    \caption{Parameter descriptions and values for the asymptomatic model with immunity.}
    \label{tab:table3}
  \end{center}
\end{table}

\pagebreak

\section*{Supplementary Figures}

\begin{figure}[!ht]
\begin{center}
\includegraphics[width=\textwidth]{figure_base_sens.ggp.pdf}
\caption{
\textbf{Simulations of a model with asymptomatic transmission and symptom-responsive transmission reduction for a wide range of asymptomatic transmissibility.}
Total deaths as a function of the proportion of asymptomatic infections $p$ across a wide range scenarios for $\delta$.
We simulate the model for 365 days, assuming $\beta_s = 0.8/\mathrm{day}$, $\nu=0.5/\mathrm{day}$, $\gamma_s=\gamma_a=0.2/\mathrm{day}$, and $f=0.01$, and an initial exposed proportion of $10^{-4}$.
We allow the ratios between the asymptomatic transmission rate $\beta_a$ and symptomatic transmission rate $\beta_s$ to vary between 0.25 and 1.
See Materials and Methods for model details and Supplementary Table S1 for parameter descriptions and values.
}
\end{center}
\end{figure}

\pagebreak

\begin{figure}[!ht]
\begin{center}
\includegraphics[width=\textwidth]{figure_base_sens_fixR0.ggp.pdf}
\caption{
\textbf{Simulations of a model with asymptomatic transmission and symptom-responsive transmission reduction for a wide range of asymptomatic transmissibility and a fixed $\mathcal{R}_0$ value at intermediate asymptomaticity.}
Total deaths as a function of the proportion of asymptomatic infections $p$ across a wide range scenarios for $\delta$.
We simulate the model for 365 days, assuming $\nu=0.5/\mathrm{day}$, $\gamma_s=\gamma_a=0.2/\mathrm{day}$, and $f=0.01$, and an initial exposed proportion of $10^{-4}$.
We allow the ratios between the asymptomatic transmission rate $\beta_a$ and symptomatic transmission rate $\beta_s$ to vary between 0.25 and 1.
We also fix the basic reproduction number $\mathcal{R}_0=4$ when there are intermediate levels of asymptomaticity $p=0.5$ and no reduction in symptomatic transmission rate $\delta=0$.
See Materials and Methods for model details and Supplementary Table S1 for parameter descriptions and values.
}
\end{center}
\end{figure}


\pagebreak

\begin{figure}[!ht]
\begin{center}
\includegraphics[width=0.8\textwidth]{diagram_sub.pdf}
\caption{
\textbf{Schematic diagram and simulations of a model with pre-symptomatic and asymptomatic transmission and symptom-responsive transmission reduction.}
(A) $S$ represents susceptible individuals; $E$ represents exposed individuals; $I_p$ represents pre-symptomatic individuals; $I_a$ represents symptomatic individuals; $I_s$ represents symptomatic individuals; $R$ represents recovered individuals; and $D$ represents deceased individuals. See Methods for model details.
(B) Total deaths as a function of the proportion of non-symptomatic transmission $\phi$ across a wide range scenarios for $\delta_s$ and proportion of non-symptomatic transmission caused by the pre-symptomatic transmission, $\eta$ (between 0\% and 100\%). 
See Materials and Methods for model details and Supplementary Table S2 for parameter descriptions and values.
}
\end{center}
\end{figure}

\pagebreak

\bibliography{main}

\end{document}
