\documentclass[12pt]{article}
\usepackage[utf8]{inputenc}

\usepackage{color}

\usepackage{xspace}

\usepackage{lmodern}
\usepackage{amssymb,amsmath}

\usepackage[pdfencoding=auto, psdextra]{hyperref}

\usepackage{natbib}
\bibliographystyle{chicago}

\newcommand{\eref}[1]{Eq.~(\ref{eq:#1})}
\newcommand{\fref}[1]{Fig.~\ref{fig:#1}}

\newcommand{\rR}{\mbox{$r$--$\cal R$}}
\newcommand{\RR}{\ensuremath{{\cal R}}}
\newcommand{\RRhat}{\ensuremath{{\hat \cal R}}}
\newcommand{\Rx}[1]{\ensuremath{{\cal R}_{#1}}} 
\newcommand{\Ro}{\ensuremath{{\mathcal R}_{0}}\xspace}
\newcommand{\Rs}{\Rx{\mathrm{s}}}
\newcommand{\Rpool}{\ensuremath{{\mathcal R}_{\textrm{\tiny{pool}}}}\xspace}
\newcommand{\Reff}{\Rx{\mathit{eff}}}
\newcommand{\Tc}{\ensuremath{C}}

\newcommand{\dd}[1]{\ensuremath{\, \mathrm{d}#1}}
\newcommand{\dtau}{\dd{\tau}}
\newcommand{\dx}{\dd{x}}
\newcommand{\dsigma}{\dd{\sigma}}

\newcommand{\rev}{\subsection*}
\newcommand{\revtext}{\textsf}
\setlength{\parskip}{\baselineskip}
\setlength{\parindent}{0em}

\newcommand{\comment}[3]{\textcolor{#1}{\textbf{[#2: }\textsl{#3}\textbf{]}}}
\newcommand{\jd}[1]{\comment{cyan}{JD}{#1}}
\newcommand{\swp}[1]{\comment{magenta}{SWP}{#1}}
\newcommand{\dc}[1]{\comment{blue}{DC}{#1}}
\newcommand{\jsw}[1]{\comment{green}{JSW}{#1}}
\newcommand{\hotcomment}[1]{\comment{red}{HOT}{#1}}

\newcommand{\psymp}{\ensuremath{p}} %% primary symptom time
\newcommand{\ssymp}{\ensuremath{s}} %% secondary symptom time
\newcommand{\pinf}{\ensuremath{\alpha_1}} %% primary infection time
\newcommand{\sinf}{\ensuremath{\alpha_2}} %% secondary infection time

\newcommand{\psize}{{\mathcal P}} %% primary cohort size
\newcommand{\ssize}{{\mathcal S}} %% secondary cohort size

\newcommand{\gtime}{\tau_{\rm g}} %% generation interval
\newcommand{\gdist}{g} %% generation-interval distribution
\newcommand{\idist}{\ell} %% incubation period distribution

\newcommand{\total}{{\mathcal T}} %% total number of serial intervals


\begin{document}

\noindent Dear Editor:

Following an editorial suggestion from PNAS, we are transferring our manuscript ``Intermediate levels of asymptomatic transmission can lead to the highest epidemic fatalities'' for consideration for publication in PNAS Nexus.
Our revised submission builds on multiple years of empirical evidence that asymptomatic transmission of SARS-CoV-2 impacts pandemic controllability and severity. 
Here, we provide and apply a significant refinement of the associated theory, linking asymptomatic transmission at individual scales to population-level epidemic outcomes.  

Consistent with PNAS reviewer suggestions, we have revised our manuscript to better highlight the theoretical nature of our work and the associated uncertainties in parametric assumptions.  Notably, we perform sensitivity analyses to show that our central conclusions are robust to parameter variation associated with asymptomatic transmissibility (Supplementary Figure S1--2). We also provide directions for future areas of research that can assess both the impact of asymptomatic transmission and the extent to which control can improve population-level outcomes. Our original conclusions remain: we find that intermediate levels of asymptomaticity lead to the highest levels of epidemic fatalities over a wide range of parameters. Hence, asymptomatic infection and transmission presents a double-edged sword. 

As context, foundational analyses of asymptomatic infections have primarily focused on the impact of asymptomatic infections on disease controllability (e.g., Fraser et al., 2004) – neglecting the effects of asymptomatic spread on population-level outcomes, such as overall fatalities. Here, we use a series of epidemic models to assess the role of asymptomatic spread on epidemic fatalities in the presence of transmission-reducing effects among symptomatic individuals (e.g., via behavioural change or symptom-based interventions). We then extend our framework to assess the ways in which immunity profiles can lead to potentially paradoxical outcomes: with less virulent strains spreading precisely because asymptomatic individuals can transmit while unaware they are infected.   Moreover, our findings show that emergence of milder variants at the individual scale can nonetheless cause similar or even higher levels of fatalities at population scales – mirroring many features of the Omicron variant. Overall, we believe that the revised work is timely, novel, and fits the PNAS Nexus mission of welcoming rigorous research with broad, interdisciplinary appeal.

Below please find our detailed responses to reviewers.

Sincerely,
Joshua S. Weitz

\pagebreak

\rev{Reviewer \#1}

\revtext{Park et al. analyzed how the level of asymptomatic transmission relates to the total fatality at the population level in an infectious disease outbreak. At an individual level, the milder the disease (e.g. asymptomatic infection) the better; however, at the population level, milder disease (everything else being equal) could lead to a higher level of transmission and thus potentially a higher fatality. The authors developed mathematical models to quantitatively understand the conditions when this tradeoff occurs, and the implications of this tradeoff. They showed that when symptomatic individuals change behavior to reduce the number of contacts and when this reduction is sufficiently large, intermediate level of asymptomatic transmission (again, everything else being equal) could lead to the highest level of fatality.}

\revtext{Under this framework, the authors further explored the impact of vaccination on fatality assuming vaccinations protect against infection, symptoms or deaths. Broadly speaking, vaccination reduces fatality as expected under most scenarios, except when vaccination protects against symptoms only and the proportion of asymptomatic infection is low. (My assessment is that these exceptions are not directly relevant to the COVID-19 pandemic, because most (or all) vaccines protect against infection, fatality and transmission to some extent and the proportion of asymptomatic infection is high). Finally, the authors used the framework to analyze the dynamics of an invading variant, and showed that when immunity induced by an earlier strain provides protection against symptoms only (without protection for infection), a future variant can cause more rapid outbreaks and a high level of fatality.}

\revtext{Overall, I enjoy reading the manuscript and the careful mathematical analyses. The work proposes a novel and interesting theoretical idea on how reducing symptomatic infection only may lead to worse outcomes at the population level. The work is very well written, and the analyses are rigorous and very thoughtful. I do not have any specific suggestions to strengthen the manuscript.}

Thank you for your comments.

\revtext{However, while this is a nicely designed and conducted piece of work and the resulting theory is interesting, the work remains theoretical in nature.}

We agree that our work remains theoretical --- and that PNAS \& PNAS Nexus are both forums for compelling theoretical work. We feel that applying this work in a broader context would be better suited as a separate paper. We have thus decided to transfer our paper to PNAS Nexus while addressing both reviewers' comments within the scope of this paper.

\revtext{Because of a lack of direct quantification of the key values in the model (e.g. $\delta$ and epsilons) from actual data, it is unclear which parameter regime a particular phase of the COVID-19 pandemic lies.}

\revtext{Therefore, despite some general principles derived from the theory, it is difficult to apply the results of the work to identify drivers of specific epidemiological patterns of SARS-CoV-2 or to make specific recommendations for intervention strategies. Therefore, I do not feel the scope of the work is broad enough for PNAS.}

\revtext{Here below I list a few reasons for my assessment.}

\revtext{1. The work first makes the assumption that symptomatic individuals change their behavior (e.g., reducing contacts or increasing mask wearing) after symptom onset, and as a result, their transmission rate is decreased by a factor (modeled using parameter $\delta$). The authors show that when $\delta$ is large (i.e. >0.6), intermediate levels of asymptomatic transmission would lead to the highest levels of fatality at the population level. This observation is the basis for a lot of nonintuitive results in this study. However, the value of $\delta$ is unknown for SARS-CoV-2 unfortunately. It is almost impossible to precisely estimate this parameter (due to the difficulties identifying all transmission events). And this parameter is likely to be very variable over time and across different countries. The authors discussed lines of indirect evidence.}

\revtext{For example, the fraction of presymptomatic transmission sets the upper bound for $\delta$. One study estimated the fraction of presymptomatic transmission to be around 30\%-60\%. Another study by some of the authors of this study suggests the pre-symptomatic transmission may only account for 20\% of transmission. This suggests that the fraction of symptomatic transmission may be as high as 80\%. The key question is whether the change of behavior change or intervention would allow symptomatic individuals to reduce transmission by 60\% (given that the upper bound is 70\% or 80\%). To reach 60\%, that would mean that most of the potential transmission events are prevented after symptom onset for all the individuals in the population.}

Thank you for pointing out a potential misunderstanding with respect to the value of $\delta$.
We previously explained that the fraction of presymptomatic transmission sets the upper bound for $\delta$ for the main model:

``For this particular model, it does not make biological sense for $\delta$ to be greater than the amount of post-symptomatic transmission, because pre-symptomatic transmission is implicitly included in the $I_s$ compartment.''

However, this is not necessarily the case for the generalized model, which includes both pre-symptomatic and asymptomatic transmission. 
To avoid confusion, we now write $\delta_s$ for the generalized model to capture the decrease in transmission only after symptom onset---the value of $\delta_s$ is independent of the fraction of presymptomatic transmission.
Throughout, we have tried to clarify our explanations and provide a concrete example:

``To further address the uncertainty in the values of $\delta$, we extend our model to include both pre-symptomatic and asymptomatic transmission.
To do so, we reparameterize the model based on the relative importance of \textit{non-symptomatic} transmission, rather than on the proportion of asymptomatic cases.
We then use $\delta_s$ to capture the decrease in transmission only after symptom onset---this means that $\delta_s$ is now independent of the amount of presymptomatic transmission.
We then fix the reproduction number of symptomatic individuals and calculate fatalities at the population level as a function of the proportion of total non-symptomatic transmission and the proportion of non-symptomatic transmission that is caused by pre-symptomatic transmission (see Materials and Methods for model details and Supplementary Table S2 for parameter descriptions and values).

Using the generalized non-symptomatic transmission model, we find a wide variety of scenarios for which peak fatalities occur at intermediate levels of non-symptomatic transmission in the presence of moderate to strong behavioral effects, $\delta_s > 0.6$ (Supplementary Figure S3; Table S2).
For example, when 40\% of non-symptomatic transmission is caused by pre-symptomatic transmission, a 60\% reduction in transmission after symptom onset is sufficient to drive the nonlinear effect of non-symptomatic transmission on epidemic fatality (peaking at around 10\% non-symptomatic transmission).
One exception is the extreme case in which all non-symptomatic transmission is caused by pre-symptomatic transmission (i.e., there is no asymptomatic transmission);
in this case, total infections and fatalities are maximized when all transmission is caused by pre-symptomatic transmission. 
While 60\% reduction in transmission after symptom onset ($\delta_s = 0.6$) is still high, it is plausible given behavioural and policy responses.
For example, \cite{kucharski2020effectiveness} estimated that up to $\sim$64\% reduction in transmission would be possible under self-isolation and self-quarantine as well as manual contact tracing (and as low as $\sim$47\% reduction using digital contact tracing without manual tracing).
Given that the majority of isolation and tracing measures likely target symptomatic transmission, the amount of reduction in symptomatic transmission would be similar to these values, providing indirect support for the feasibility of high $\delta_s$ values. 
Hereafter, we focus on asymptomatic infections for simplicity, but our conclusions have implications for the more general case of non-symptomatic transmission.
We return to the discussion of values of $\delta$ later in the Discussion section.''

\revtext{This seems to be unlikely given other indirect data, e.g. the relatively high household secondary attack rate (SAR) for symptomatic individuals (e.g. see Madewell et al. JAMA Netw Open. 2022;5(4):e229317.). Of course, the SAR data does not completely rule out the possibility that $\delta>0.6$; however, the different indirect datasets highlight the difficulty in estimating $\delta$ and that the evidence for large $\delta$ is unclear.}

We have further tried to highlight the uncertainty in the values of $\delta$:

``The main conclusion of our analysis relies on having high levels of transmission reduction among symptomatic individuals ($\delta$). 
For the simple model, the value of $\delta$ is limited by the amount of presymptomatic transmission, which makes high $\delta$ values unrealistic for COVID-19.
Therefore, we extended our model to include both presymptomatic and asymptomatic transmission to show that our results hold for a more general case: for high levels of transmission reduction after symptom onset ($\delta_s$), epidemic fatalities peak at intermediate values of non-symptomatic transmission.
Based on early estimates for the contact tracing effectiveness \citep{kucharski2020effectiveness} high levels are $\delta_s$ are plausible, but uncertainty remains. 
Furthermore, values of $\delta$ (and likewise, $\delta_s$) likely change over the course of an epidemic.
During the exponential growth phase, $\delta$ is likely low because there is limited awareness for the outbreak and intervention measures in place.
As the number of cases, hospitalizations, and fatalities increase, $\delta$ will also increase, reflecting changes in awareness-driven behavior and intensity of non-pharmaceutical interventions \citep{weitz2020awareness}.
Further analysis is needed to constrain the uncertainty in $\delta$ and to apply the model framework in distinct disease, policy, and socioeconomic contexts. ''

\revtext{2. The results in Fig. 3 and 4 are very interesting. They showed under certain parameter regimes, vaccination (that prevents symptoms only) or a milder variant may not lead to lower fatality at the population level. Again, this result is under the assumption that $\delta>0.6$. The results could be useful for vaccine design and understanding variant dynamics.}

Thank you.

\revtext{However, as mentioned earlier, most (or all) vaccines most (or all) vaccines protect against infection, fatality and transmission to some extent. In addition, without thorough analyses of experimental or epidemiological data to estimate the relevant parameters for SARS-CoV-2 variants (e.g. Delta vs. Omicron as mentioned by the authors), it is unclear how the theory relates to the observed impacts of vaccination and the selection of variants on SARS-CoV-2 dynamics.} 

We have tried to highlight the uncertainty in parameters and robustness of results to parameter uncertainty throughout the paper. 
We have also added the following paragraph in the discussion section to provide guidance for future work:

``Our work uses theoretical model to illustrate the potential for asymptomatic infection and transmission to make population-level outcomes worse.
Applying these ideas to specific outbreak scenarios will require narrowing down uncertainties in key parameters, such as the degree of symptom-responsive transmission reduction and immunity profiles.
For example, we simulated unmitigated outbreaks with fixed parameters but epidemic dynamics, especially those of SARS-CoV-2, are more complex, reflecting changes in intervention efforts and the emergence of new variants.
Calibrating the model to outbreak data is therefore critical to understanding the role of asymptomatic transmission across different epidemic phases.
We also showed that asymptomatic infections can have important implications for evolutionary dynamics, but their contributions in driving evolutionary dynamics of SARS-CoV-2 is yet unclear.
Our study provides a starting point for exploring these questions in more detail.''

\revtext{Overall, the work is a very interesting and well-conducted theoretical work; however, its utility in understanding SARS-CoV-2 epidemiology or design of intervention strategies is limited.}

We hope that our revisions clarify the generalized nature of our findings and also point the way towards practical steps to link individual and population-level outcomes associated with asymptomatic transmission in realistic contexts. 


\rev{Reviewer \#2}

\revtext{Park et al present a very well developed epidemiological model focused on the interplay between asymptomatic transmission and symptom-responsive transmission reduction due to behavioral change or mitigation. The model is well developed and the analysis rigorous, in as far as it goes and overall everything is very well described and written. I found the key predictions - that intermediate asymmetric (and pre-symptomatic) transmission has the greatest population level impact when there is mitigation and behavioral change for symptomatic transmission and all of the outcomes in the abstract to be pretty intuitive. The detailed analysis in the main paper are very useful and the paper is at its most interesting when it is making more specific predictions.}

\revtext{I would focus on the level of transmission blocking that is important to mitigate the population level effects of asymptomatic spread, rather than simply stating that modest effects can do this.}

\revtext{My sense is that it is not at all surprising that anti-disease immunity CAN lead to greater population level effects, what is more interesting is when it does. In particular, this could be related to the specific challenges of the Sars-Cov-2 pandemic. The authors have the models to easily get these results and this refocusing will improve the impact of the paper. The discussion of the implications of the model to our understanding of new variants and in particular Omicron were very interesting and informative. Overall, I think that the manuscript would have a greater impact with a rewrite that focusses less on the general predictions, since these are not surprising and more on the detailed drivers of the results and how these are illustrated by SARS-Cov-2. I have some specific criticisms.}

We appreciate the suggestion of a rewrite but note that we are unaware of another paper that makes the explicit link between intermediate levels of asymptomatic transmission and worst-case population-level outcomes.  
The fact that our models are simple to understand and 'easily' get these results seems to us to be a positive reflection of our choice of a minimal set of features that nonetheless lead to a robust gap between individual and population level outcomes.  
We hope that this model embodies the spirit of 'Everything is obvious: once you know the answer' - the book on discovery and common sense by Duncan Watts from >10 years ago. 
Turning back to early 2020, it was certainly not clear to (most) policy makers and (most) models that asymptomatic transmission and control could be a significant driver of improving population level outcomes.  
Hence, we have decided to leave the core structure in place. 
In addition, we feel that making detailed predictions are beyond the scope of this paper. 

While immunity profiles have been discussed in other contexts (such as the Gandon et al paper as suggested by the same reviewer), there has been a limited discussion around the topic from epidemiological and public health perspectives.
Therefore, we wanted to focus on qualitative effects of asymptomatic transmission on epidemic fatalities in this paper, while leaving quantitative predictions for the future work.
Throughout the paper, we have tried to highlight the theoretical nature of our work and give guidance for future work.
In particular, we have added the following paragraph in the discussion section:

``Our work uses theoretical model to illustrate the potential for asymptomatic infection and transmission to make population-level outcomes worse.
Applying these ideas to specific outbreak scenarios will require narrowing down uncertainties in key parameters, such as the degree of symptom-responsive transmission reduction and immunity profiles.
For example, we simulated unmitigated outbreaks with fixed parameters but epidemic dynamics, especially those of SARS-CoV-2, are more complex, reflecting changes in intervention efforts and the emergence of new variants.
Calibrating the model to outbreak data is therefore critical to understanding the role of asymptomatic transmission across different epidemic phases.
We also showed that asymptomatic infections can have important implications for evolutionary dynamics, but their contributions in driving evolutionary dynamics of SARS-CoV-2 is yet unclear.
Our study provides a starting point for exploring these questions in more detail.''

\revtext{My main question about the results comes from the assumption on Line 117 - how important is the 25\% reduction in asymptomatic transmission assumption? - I was surprised that given the uncertainty that they acknowledge, the authors didn't explore this. My sense is that this is a high assumption for asymptomatic transmission, but some exploration of the importance of this assumption to the outcomes is important. Clearly, this will not impact the general predictions, but if they focus the paper on more specific scenarios, it may have an impact and to my mind needs to be explored.}

We now perform sensitivity analyses to account for this uncertainty:

``We note that these results are robust to uncertainties in asymptomatic transmission rate.
In Supplementary Figure S2, we perform the same analysis while varying the ratio between $\RR_a$ and $\RR_s$ between 0.25 and 1.
In this case, qualitative predictions are robust when $\RR_a \geq 0.5 \RR_s$
--- when $\RR_a = 0.25 \RR_s$ and $\delta$ is large then $\RR_0 < 1$ and the epidemic does not take off.
In Supplementary Figure S3, we perform the same sensitivity analysis as Supplementary Figure S2 while fixing $\RR_0 = 4$ when $p=0.5$ and $\delta=0$ to prevent the epidemic from dying out. In this case, we find that intermediate values of asymptomaticity lead to the highest epidemic fatalities at high $\delta$ values across all ranges of $\RR_a/\RR_s$ we consider.''

\revtext{As a general criticism, the focus on immunity profiles and whether immunity is symptom blocking is interesting and novel in this context, but it is not surprising for reviewers who are familiar with the evolution of tolerance literature. In particular, there should be a better link to the Gandon et al paper on the impact of anti-disease vaccines on epidemiological as well as evolutionary outcomes. Clearly the focus of the Gandon paper is different, but they have comparable insights of the impact on epidemiology of anti-diseases (symptoms) immunity compared to transmission blocking ones. I think this literature should be acknowledged, not least because it makes many of the same predictions.}

We now discuss similarities and differences between our work and Gandon et al paper:

``Our conclusions echo earlier findings by \cite{gandon2001imperfect}, who showed that vaccine-derived immunity against diseases can promote the evolution of more virulent strains in unvaccinated individuals.
By allowing explicit parasite evolution, they also showed that total malaria mortality peaks at intermediate levels of vaccination; our analysis reveals that the differences in symptomatic and asymptomatic transmission behavior can give rise to a similar effect even in the absence of evolution.  ''

\revtext{Line 58 - "subclinical or even asymptomatic' is imprecise and confusing}

Thank you for pointing this out. We have removed ``or even asymptomatic''

\revtext{Paragraph at line 59 - is the diamond princess data still the best data we have? It would be useful to emphasize that the asymptomatic transmission is very well established.}

Since we are focusing on the asymptomatic transmission of primary infection, we do feel that the diamond princess data is one of the best data we have. We have tried to clarify this point while addressing other existing data on asymptomatic transmission: 

``It is now well established that asymptomatic individuals can transmit SARS-CoV-2 infections \citep{gao2021role,johansson2021sars,subramanian2021quantifying,koelle2022changing,lizewski2022navy}, but asymptomatic cases are increasingly shaped by prior immunity (whether through infection, vaccination, or both).
In contrast, early in the pandemic, a COVID-19 outbreak on the Diamond Princess cruise ship played a critical role in understanding the role of asymptomatic infections in the spread of SARS-CoV-2 from and to immunologically naive individuals ...''

\revtext{Paragraph line 69 - It would be useful to discuss how important the symptoms may be for transmission here - coughing, sneezing etc may of course have a direct impact on transmission and it can be this rather than differences in viral loads are likely to be important.}

Thank you for pointing this out. We added the following:

``The likely role of symptoms in transmission further contributes to this uncertainty.
Coughing and sneezing can help deliver virus-containing droplets. However, the prevalence of presymptomatic and asymptomatic transmission of SARS-CoV-2 suggests that speech droplets can also be an important mode of transmission \citep{stadnytskyi2021breathing}.''

\bibliography{main}

\end{document}
